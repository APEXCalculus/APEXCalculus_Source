\thispagestyle{empty}
\Huge
\noindent {\bf \textsc{Preface}}\\
\large
\emph{A Note on Using this Text}
\vspace{1in}
\normalsize

Thank you for reading this short preface. Allow us to share a few key points about the text so that you may better understand what you will find beyond this page.

This text comprises a three--text series on Calculus. This first part covers material taught in many ``Calc 1'' courses: limits, derivatives, and the basics of integration. The second text covers material often taught in ``Calc 2:'' integration and its applications, along with an introduction to sequences, series and Taylor Polynomials. The third text, which is not yet fully written, covers topics common in ``Calc 3'' or ``multivariable calc:'' parametric equations, polar coordinates, vector--valued functions, and functions of more than one variable. All three are available separately for free at \texttt{\href{http://www.vmi.edu/APEX}{www.vmi.edu/APEX}}. %These three texts are intended to work together and make one cohesive text, \textit{APEX Calculus}, which can also be downloaded from the website. 

Printing the entire text as one volume makes for a large, heavy, cumbersome book. One can certainly only print the pages they currently need, but some prefer to have a nice, bound copy of the text. Therefore this text has been split into these three manageable parts, each of which can be purchased for about \$7 at \href{http://amazon.com}{Amazon.com}.\\ 

%A result of this splitting is that sometimes a concept is said to be explored in a ``later section,'' though that section does not actually appear in this particular text. Downloading the .pdf of \textit{APEX Calculus} will ensure that you have all the content.  
%material is referenced that is not contained in the present text. The context should make it clear whether the ``missing'' material is in the \textit{Calculus I} or \textit{Calculus III} portion. Downloading the appropriate .pdf, or the whole \textit{APEX Calculus} .pdf, will give access to these topics.
% This splitting of the material also results in unfortunate page/chapter numberings. Chapter 5 of this text is Chapter 1 of \textit{Calculus II}. Apart from these numberings, page--for--page the content of the sections that appear in both \textit{Calculus I} and \textit{Calculus II} are identical.\\ %For instance, in this text, ``Theorem 20'' may be mentioned, although Theorem 20 is only presented in Part I. To minimize confusion, theorems, definitions and key ideas are referenced by their title or subject matter, not their number.

%The current publisher of this text does not allow one text to be split across multiple volumes, with continuity of chapters and page numberings. This is the one drawback of the current publishing model that has many advantages, highlighted below. Because of this, there are a few peculiarities 

\noindent\textbf{\large \apex\  -- Affordable Print and Electronic teXts}\\

\apex\ is a consortium of authors  who collaborate to produce high--quality, low--cost textbooks. The current textbook--writing paradigm is facing a potential revolution as desktop publishing and electronic formats increase in popularity. However, writing a good textbook is no easy task, as the time requirements alone are substantial. It takes countless hours of work to produce text, write examples and exercises, edit and publish. Through collaboration, however, the cost to any individual can be lessened, allowing us to create texts that we freely distribute electronically and sell in printed form for an incredibly low cost. Having said that, nothing is entirely free; someone always bears some cost. This text ``cost'' the authors of this book their time, and that was not enough. \textit{APEX Calculus} would not exist had not the Virginia Military Institute, through a generous Jackson--Hope grant, given one of the authors significant time away from teaching so he could focus on this text.

Each text is available as a free .pdf, protected by a Creative Commons Attribution - Noncommercial 3.0 copyright. That  means you can give the .pdf to anyone you like, print it in any form you like, and even edit the original content and redistribute it. If you do the latter, you must  clearly reference this work and you cannot sell your edited work for money.

We encourage others to adapt this work to fit their own needs. One might add sections that are ``missing'' or remove sections that your students won't need. Please contact the authors if you are interested and they will make their source files available to you.

You can learn more at \texttt{\href{http://www.vmi.edu/APEX}{www.vmi.edu/APEX}}.
\thispagestyle{empty}


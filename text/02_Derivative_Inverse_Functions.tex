\section{Derivatives of Inverse Functions}\label{sec:deriv_inverse_function}

Recall that a function $y=f(x)$ is said to be \textit{one to one} if it passes the horizontal line test; that is, for two different $x$ values $x_1$ and $x_2$, we do \textit{not} have $f(x_1)=f(x_2)$. In some cases the domain of $f$ must be restricted so that it is one to one. For instance, consider $f(x)=x^2$. Clearly, $f(-1)= f(1)$, so $f$ is not one to one on its regular domain, but by restricting $f$ to $(0,\infty)$, $f$ is one to one.\index{derivative!inverse function}

Now recall that one to one functions have \textit{inverses}. That is, if $f$ is one to one, it has an inverse function, denoted by $f\primeskip^{-1}$, such that if $f(a)=b$, then $f\primeskip^{-1}(b) = a$. The domain of $f\primeskip^{-1}$ is the range of $f$, and vice-versa. For ease of notation, we set $g=f\primeskip^{-1}$ and treat $g$ as a function of $x$.

Since $f(a)=b$ implies $g(b)=a$, when we compose $f$ and $g$ we get a nice result: $$f\big(g(b)\big) = f(a) = b.$$ In general, $f\big(g(x)\big) =x$ and $g\big(f(x)\big) = x$. This gives us a convenient way to check if two functions are inverses of each other: compose them and if the result is $x$, then they are inverses (on the appropriate domains.)

When the point $(a,b)$ lies on the graph of $f$, the point $(b,a)$ lies on the graph of $g$. This leads us to discover that the graph of $g$ is the reflection of $f$ across the line $y=x$. In Figure \ref{fig:inverse1} we see a function graphed along with its inverse. See how the point $(1,1.5)$ lies on one graph, whereas $(1.5,1)$ lies on the other. Because of this relationship, whatever we know about $f$ can quickly be transferred into knowledge about $g$.

\mfigure{.6}{A function $f$ along with its inverse $f\primeskip^{-1}$. (Note how it does not matter which function we refer to as $f$; the other is $f\primeskip^{-1}$.)}{fig:inverse1}{figures/figinverse1}

For example, consider Figure \ref{fig:inverse2} where the tangent line to $f$ at the point $(a,b)$ is drawn. That line has slope $\fp(a)$. Through reflection across $y=x$, we can see that the tangent line to $g$ at the point $(b,a)$ should have slope $\ds \frac{1}{\fp(a)}$. This then tells us that $\ds g\primeskip'(b) = \frac{1}{\fp(a)}.$

\mfigure{.35}{Corresponding tangent lines drawn to $f$ and $f\primeskip^{-1}$.}{fig:inverse2}{figures/figinverse2}

Consider:
\begin{center}
	\begin{tabular}{ccc}
	Information about $f$ & & Information about $g=f\primeskip^{-1}$ \\ \hline
	\parbox{100pt}{\centering $(-0.5,0.375)$ lies on $f$}\rule{0pt}{12pt} & \hskip 40pt & \parbox{100pt}{\centering $(0.375,-0.5)$ lies on $g$}\\
	\rule{0pt}{20pt}\parbox{100pt}{\centering Slope of tangent line to $f$ at $x=-0.5$ is $3/4$} & & \parbox{100pt}{\centering Slope of tangent line to $g$ at $x=0.375$ is $4/3$}\rule{0pt}{17pt} \\
	\rule{0pt}{15pt}$\fp(-0.5) = 3/4$ & & $g\primeskip'(0.375) = 4/3$\rule{0pt}{12pt}
	\end{tabular}
\end{center}

We have discovered a relationship between $\fp$ and $g\primeskip'$ in a mostly graphical way. We can realize this relationship analytically as well. Let $y = g(x)$, where again $g = f\primeskip^{-1}$. We want to find $\ds y\primeskip'$. Since $y = g(x)$, we know that $f(y) = x$. Using the Chain Rule and Implicit Differentiation, take the derivative of both sides of this last equality.
		\begin{align*}
			\frac{d}{dx}\Big(f(y)\Big) &= \frac{d}{dx}\Big(x\Big) \\
			\fp(y)\cdot y\primeskip' &= 1\\
			y\primeskip' &= \frac{1}{\fp(y)}\\
			y\primeskip' &= \frac{1}{\fp(g(x))}
		\end{align*}
		
This leads us to the following theorem.

\theorem{thm:deriv_inverse_functions}{Derivatives of Inverse Functions}
{Let $f$ be differentiable and one to one on an open interval $I$, where $\fp(x) \neq 0$ for all $x$ in $I$, let $J$ be the range of $f$ on $I$, let $g$ be the inverse function of $f$, and let $f(a) = b$ for some $a$ in $I$. Then $g$ is a differentiable function on $J$, and in particular,
	
%	\begin{center}
\hskip-7pt	\begin{tabular}{ccc}
	1. $\ds \left(f\primeskip^{-1}\right)'(b)=g\primeskip'(b) = \frac{1}{\fp(a)}$ &\hskip 4pt and \hskip 4pt&  2. $\ds \left(f\primeskip^{-1}\right)'(x)=g\primeskip'(x) = \frac{1}{\fp(g(x))}$
	\end{tabular}
%	\end{center}
}

The results of Theorem \ref{thm:deriv_inverse_functions} are not trivial; the notation may seem confusing at first. Careful consideration, along with examples, should earn understanding.

In the next example we apply Theorem \ref{thm:deriv_inverse_functions} to the arcsine function.\\

\example{ex_deriv_arcsin}{Finding the derivative of an inverse trigonometric function}{
Let $y = \arcsin x = \sin^{-1} x$. Find $y\primeskip'$ using Theorem \ref{thm:deriv_inverse_functions}.}
{Adopting our previously defined notation, let $g(x) = \arcsin x$ and $f(x) = \sin x$. Thus $\fp(x) = \cos x$. Applying the theorem, we have 
			\begin{align*}
			g\primeskip'(x) &= \frac{1}{\fp(g(x))} \\
						&= \frac{1}{\cos(\arcsin x)}.
			\end{align*}
			
This last expression is not immediately illuminating. Drawing a figure will help, as shown in Figure \ref{fig:inverse3}. Recall that the sine function can be viewed as taking in an angle and returning a ratio of sides of a right triangle, specifically, the ratio ``opposite over hypotenuse.'' This means that the arcsine function takes as input a ratio of sides and returns an angle. The equation $y=\arcsin x$ can be rewritten as $y=\arcsin (x/1)$; that is, consider a right triangle where the hypotenuse has length 1 and the side opposite of the angle with measure $y$ has length $x$. This means the final side has length $\sqrt{1-x^2}$, using the Pythagorean Theorem.

\vskip \baselineskip
\mfigure{.8}{A right triangle defined by $y=\sin ^{-1}(x/1)$ with the length of the third leg found using the Pythagorean Theorem.}{fig:inverse3}{figures/figinverse3}
\vskip \baselineskip

Therefore $\cos (\sin^{-1} x) = \cos y = \sqrt{1-x^2}/1 = \sqrt{1-x^2}$, resulting in $$\frac{d}{dx}\big(\arcsin x\big) = g\primeskip'(x) = \frac{1}{\sqrt{1-x^2}}.$$
\vskip-\baselineskip
}\\

Remember that the input $x$ of the arcsine function is a ratio of a side of a right triangle to its hypotenuse; the absolute value of this ratio will never be greater than 1. Therefore the inside of the square root will never be negative.

In order to make $y=\sin x$ one to one, we restrict its domain to $[-\pi/2,\pi/2]$; on this domain, the range is $[-1,1]$. Therefore the domain of $y=\arcsin x$ is $[-1,1]$ and the range is $[-\pi/2,\pi/2]$. When $x=\pm 1$, note how the derivative of the arcsine function is undefined; this corresponds to the fact that as $x\to \pm1$, the tangent lines to arcsine approach vertical lines with undefined slopes.


\mfigure{.5}{Graphs of $\sin x$ and $\sin^{-1}x$ along with corresponding tangent lines.}{fig:inverse4}{figures/figinverse4}

%\mfigure{.35}{Graphs of $\sin x$ and $\sin^{-1}x$}{fig:inverse4b}{figures/figinverse4}

In Figure \ref{fig:inverse4} we see $f(x) = \sin x$ and $f\primeskip^{-1} = \sin^{-1} x$ graphed on their respective domains. The line tangent to $\sin x$ at the point $(\pi/3, \sqrt{3}/2)$ has slope $\cos \pi/3 = 1/2$. The slope of the corresponding point on $\sin^{-1}x$, the point $(\sqrt{3}/2,\pi/3)$, is $$\frac{1}{\sqrt{1-(\sqrt{3}/2)^2}} = \frac{1}{\sqrt{1-3/4}} = \frac{1}{\sqrt{1/4}} = \frac{1}{1/2}=2,$$ verifying yet again that at corresponding points, a function and its inverse have reciprocal slopes.\\

Using similar techniques, we can find the derivatives of all the inverse trigonometric functions. In Figure \ref{fig:domain_trig} we show the restrictions of the domains of the standard trigonometric functions that allow them to be invertible.\\

\noindent\hskip-110pt%
\noindent\begin{minipage}{\textwidth+200pt}
\small\noindent
%\centering%\begin{center}
%\noindent\begin{minipage}[t]{.5\textwidth}%
\begin{tabular}{cccccc}
Function & Domain & Range &\parbox[b]{40pt}{\centering Inverse Function} & Domain & Range\\ \hline
\rule{0pt}{12pt} $\sin x$ & $[-\pi/2, \pi/2]$ & $[-1,1]$&$\sin^{-1} x$ & $[-1,1]$ & $[-\pi/2, \pi/2]$ \\
\rule{0pt}{12pt}$\cos x$ & $[0,\pi]$ & $[-1,1]$&$\cos^{-1}(x)$ & $[-1,1]$ & $[0,\pi]$ \\
\rule{0pt}{12pt}$\tan x$ & $(-\pi/2,\pi/2)$ & $(-\infty,\infty)$&$\tan^{-1}(x)$ & $(-\infty,\infty)$ & $(-\pi/2,\pi/2)$	\\
\rule{0pt}{12pt} $\csc x$ & $[-\pi/2,0)\cup (0, \pi/2]$ & $(-\infty,-1]\cup [1,\infty)$&$\csc^{-1} x$ & $(-\infty,-1]\cup [1,\infty)$ & $[-\pi/2,0)\cup (0, \pi/2]$  \\
\rule{0pt}{12pt}$\sec x$ & $[0,\pi/2)\cup (\pi/2,\pi]$ & $(-\infty,-1]\cup [1,\infty)$&$\sec^{-1}(x)$ & $(-\infty,-1]\cup [1,\infty)$ & $[0,\pi/2)\cup (\pi/2,\pi]$ \\
\rule{0pt}{12pt}$\cot x$ & $(0,\pi)$ & $(-\infty,\infty)$&$\cot^{-1}(x)$ &  $ (-\infty,\infty)$ & $(0,\pi)$	
\end{tabular}
\captionsetup{type=figure}
\caption{Domains and ranges of the trigonometric and inverse trigonometric functions.}\label{fig:domain_trig}
%\end{center}
%\normalsize
\end{minipage}
%\captionsetup{type=figure}%
%\caption{Domains and ranges of the trigonometric and inverse trigonometric functions.}\label{fig:domain_trig}
%\end{minipage}
%}

%%%%%%%%%%%%%%%%%%%%
%%%%%%%%%%%%%%%%%%%%
%%% The following is for preservation for a longpage edition sometime
%%%%%%%%%%%%%%%%%%%%
%%\begin{tabular}{ccc}
%%Function & Domain & Range \\ \hline
%%\rule{0pt}{12pt} $\sin x$ & $[-\pi/2, \pi/2]$ & $[-1,1]$ \\
%%\rule{0pt}{12pt}$\cos x$ & $[0,\pi]$ & $[-1,1]$ \\
%%\rule{0pt}{12pt}$\tan x$ & $(-\pi/2,\pi/2)$ & $(-\infty,\infty)$	\\
%%\rule{0pt}{12pt} $\csc x$ & $[-\pi/2,0)\cup (0, \pi/2]$ & $(-\infty,-1]\cup [1,\infty)$ \\
%%\rule{0pt}{12pt}$\sec x$ & $[0,\pi/2)\cup (\pi/2,\pi]$ & $(-\infty,-1]\cup [1,\infty)$ \\
%%\rule{0pt}{12pt}$\cot x$ & $(0,\pi)$ & $(-\infty,\infty)$	
%%\end{tabular}
%%%\end{minipage}
%%%\begin{minipage}[t]{.5\textwidth}
%%%\vskip \baselineskip
%%\hskip 10pt
%%\begin{tabular}{ccc}
%%\parbox[b]{40pt}{\centering Inverse Function} & Domain & Range \\ \hline
%%\rule{0pt}{12pt} $\sin^{-1} x$ & $[-1,1]$ & $[-\pi/2, \pi/2]$ \\
%%\rule{0pt}{12pt}$\cos^{-1}(x)$ & $[-1,1]$ & $[0,\pi]$\\
%%\rule{0pt}{12pt}$\tan^{-1}(x)$ & $(-\infty,\infty)$ & $(-\pi/2,\pi/2)$\\
%%\rule{0pt}{12pt} $\csc^{-1} x$ & $(-\infty,-1]\cup [1,\infty)$ & $[-\pi/2,0)\cup (0, \pi/2]$  \\
%%\rule{0pt}{12pt}$\sec^{-1}(x)$ & $(-\infty,-1]\cup [1,\infty)$ & $[0,\pi/2)\cup (\pi/2,\pi]$\\
%%\rule{0pt}{12pt}$\cot^{-1}(x)$ &  $ (-\infty,\infty)$ & $(0,\pi)$
%%\end{tabular}



\theorem{thm:deriv_inverse_trig}{Derivatives of Inverse Trigonometric Functions}
{The inverse trigonometric functions are differentiable on all open sets contained in their domains (as listed in Figure \ref{fig:domain_trig}) and their derivatives are as follows:\\

\noindent	\begin{minipage}{.5\specialboxlength}\small
	\begin{enumerate}
	\item		$\ds \frac{d}{dx}\big(\sin^{-1}(x)\big) = \frac{1}{\sqrt{1-x^2}}$ 
	\item		$\ds \frac{d}{dx}\big(\sec^{-1}(x)\big) = \frac{1}{|x|\sqrt{x^2-1}}$
	\item		$\ds \frac{d}{dx}\big(\tan^{-1}(x)\big) = \frac{1}{1+x^2}$
	\end{enumerate}
	\end{minipage}
	\begin{minipage}{.5\specialboxlength}\small
	\begin{enumerate}\addtocounter{enumi}{3}
	\item		$\ds \frac{d}{dx}\big(\cos^{-1}(x)\big) = -\frac{1}{\sqrt{1-x^2}}$ 
	\item		$\ds \frac{d}{dx}\big(\csc^{-1}(x)\big) = -\frac{1}{|x|\sqrt{x^2-1}}$
	\item		$\ds \frac{d}{dx}\big(\cot^{-1}(x)\big) = -\frac{1}{1+x^2}$
	\end{enumerate}\index{derivative!inverse trig.}
	\normalsize
	\end{minipage}
}			

Note how the last three derivatives are merely the opposites of the first three, respectively. Because of this, the first three are used almost exclusively throughout this text.\\

In Section \ref{sec:basic_diff_rules}, we stated without proof or explanation that $\ds \frac{d}{dx}\big(\ln x\big) = \frac{1}{x}$. We can justify that now using Theorem \ref{thm:deriv_inverse_functions}, as shown in the example.\\

\example{ex_deriv_lnx}{Finding the derivative of $y=\ln x$}{
Use Theorem \ref{thm:deriv_inverse_functions} to compute $\ds \frac{d}{dx}\big(\ln x\big)$.}
{View $y= \ln x$ as the inverse of $y = e^x$. Therefore, using our standard notation, let $f(x) = e^x$ and $g(x) = \ln x$. We wish to find $g\primeskip'(x)$. Theorem \ref{thm:deriv_inverse_functions} gives:
		\begin{align*}
		g\primeskip'(x) &= \frac{1}{\fp(g(x))} \\
					&=	\frac{1}{e^{\ln x}}\rule{0pt}{15pt} \\
					&= \frac{1}{x}.\rule{0pt}{17pt}
		\end{align*}
\vskip-\baselineskip
}\\

%\vskip\baselineskip

In this chapter we have defined the derivative, given rules to facilitate its computation, and given the derivatives of a number of standard functions. We restate the most important of these in the following theorem, intended to be a reference for further work.

\theorem{thm:deriv_glossary}{Glossary of Derivatives of Elementary Functions}
{Let $u$ and $v$ be differentiable functions, and let $a$, $c$ and $n$ be real numbers, $a>0$, $n\neq 0$. \\

\noindent%
	\begin{minipage}{.5\specialboxlength}
	\begin{enumerate}
	\item		$\frac{d}{dx}\big(cu\big) = cu'$\addtocounter{enumi}{1}
	\item		$\frac{d}{dx}\big(u\cdot v\big) = uv'+u'v$\addtocounter{enumi}{1}
	\item		$\frac{d}{dx}\big(u(v)\big) = u'(v)v'$\addtocounter{enumi}{1}
	\item		$\frac{d}{dx}\big(x\big) = 1$\addtocounter{enumi}{1}
	\item		$\frac{d}{dx}\big(e^x\big) = e^x$\addtocounter{enumi}{1}
	\item		$\frac{d}{dx}\big(\ln x\big) = \frac{1}{x}$\addtocounter{enumi}{1}
	\item		$\frac{d}{dx}\big(\sin x\big) = \cos x$\addtocounter{enumi}{1}
	\item		$\frac{d}{dx}\big(\csc x\big) = -\csc x\cot x$\addtocounter{enumi}{1}
	\item		$\frac{d}{dx}\big(\tan x\big) = \sec^2x$\addtocounter{enumi}{1}
	\item		$\frac{d}{dx}\big(\sin^{-1}x\big) = \frac{1}{\sqrt{1-x^2}}$\addtocounter{enumi}{1}
	\item		$\frac{d}{dx}\big(\csc^{-1}x\big) = -\frac{1}{|x|\sqrt{x^2-1}}$\addtocounter{enumi}{1}
	\item		$\frac{d}{dx}\big(\tan^{-1}x\big) = \frac{1}{1+x^2}$\addtocounter{enumi}{1}
	\end{enumerate}
\normalsize
\end{minipage}
\begin{minipage}{.5\specialboxlength}
	\begin{enumerate}\addtocounter{enumi}{1}
	\item		$\frac{d}{dx}\big(u\pm v\big) = u'\pm v'$\addtocounter{enumi}{1}
	\item		$\frac{d}{dx}\big(\frac uv\big) = \frac{u'v-uv'}{v^2}$\addtocounter{enumi}{1}
	\item		$\frac{d}{dx}\big(c\big) = 0$\addtocounter{enumi}{1}
	\item		$\frac{d}{dx}\big(x^n\big) = nx^{n-1}$\addtocounter{enumi}{1}
	\item		$\frac{d}{dx}\big(a^x\big) = \ln a\cdot a^x$\addtocounter{enumi}{1}
	\item		$\frac{d}{dx}\big(\log_a x\big) = \frac{1}{\ln a}\cdot\frac{1}{x}$\addtocounter{enumi}{1}
	\item		$\frac{d}{dx}\big(\cos x\big) = -\sin x$\addtocounter{enumi}{1}
	\item		$\frac{d}{dx}\big(\sec x\big) = \sec x\tan x$\addtocounter{enumi}{1}
	\item		$\frac{d}{dx}\big(\cot x\big) = -\csc^2x$\addtocounter{enumi}{1}
	\item		$\frac{d}{dx}\big(\cos^{-1}x\big) = -\frac{1}{\sqrt{1-x^2}}$\addtocounter{enumi}{1}
	\item		$\frac{d}{dx}\big(\sec^{-1}x\big) = \frac{1}{|x|\sqrt{x^2-1}}$\addtocounter{enumi}{1}
	\item		$\frac{d}{dx}\big(\cot^{-1}x\big) = -\frac{1}{1+x^2}$
	\end{enumerate}
\normalsize
\end{minipage}
}

\printexercises{exercises/02_07_exercises}

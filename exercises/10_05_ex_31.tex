{Let lines $\ell_1(t)$ and $\ell_2(t)$ be parallel. Show why the distance formula cannot be used as stated to find the distance between the lines, then show why letting $c=(\vv{P_1P_2}\times\vec d_2)\times\vec d_2$ allows one to the use the given formula.
}
{The distance formula cannot be used because since $\vec d_1$ and $\vec d_2$ are parallel, $\vec c$ is $\vec 0$ and we cannot divide by $\vnorm{0}$.

Since $\vec d_1$ and $\vec d_2$ are parallel, $\vv{P_1P_2}$ lies in the plane formed by the two lines. Thus $\vv{P_1P_2}\times\vec d_2$ is orthogonal to this plane, and $\vec c=(\vv{P_1P_2}\times\vec d_2)\times \vec d_2$ is parallel to the plane, but still orthogonal to both $\vec d_1$ and $\vec d_2$. We desire the length of the projection of $\vv{P_1P_2}$ onto $\vec c$, which is what the formula provides.
}


{$\vrt = \la t,1/t \ra$ on $(0,4]$; consider $t=1$ and $t=2$.
}
{$a_{\text{T}} = \frac{-2/t^5}{\sqrt{1+1/t^4}}$ and $a_{\text{N}} = \sqrt{\frac{4}{t^6}-\frac{4/t^{10}}{1+1/t^4}}$\\
At $t=1$, $a_{\text{T}} = \sqrt{2}$ and $a_{\text{N}} = -\sqrt{2}$;\\
At $t=2$, $a_{\text{T}} = -\frac{1}{4\sqrt{17}}$ and $a_{\text{N}} = \frac{1}{\sqrt{17}}$.\\
At $t=1$, acceleration comes from changing speed and changing direction in ``equal measure;'' at $t=2$, acceleration is nearly $\vec 0$ as it is; the low value of $a_{\text{T}}$ shows that the speed is nearly constant and the low value of $a_{\text{N}}$ shows the direction is not changing quickly.
}
{$\ds f(x,y) = 3y-2x^2$, constrained to the region bounded by the parabola $y=x^2+x-1$ and the line $y=x$.
}
{The region has two ``corners'' at $(-1,-1)$ and $(1,1)$.\\
Along the line $y=x$, $f(x,y)$ becomes $f(x) = 3x-2x^2$. Along this line, we have a critical point at $(3/4,3/4)$.\\
Along the curve $y=x^2+x-1$, $f(x,y)$ becomes $f(x) =x^2+3x-3$. There is a critical point along this curve at $(-3/2, -1/4)$. Since $x=-3/2$ lies outside our bounded region, we ignore this critical point.\\
The function $f$ itself has no critical points. \\
Checking the value of $f$ at $(-1,-1)$, $(1,1)$, $(3/4,3/4)$, we find the absolute maximum is at $(3/4,3/4,9/8)$ and the absolute minimum is at $(-1,-1,-5)$.
}
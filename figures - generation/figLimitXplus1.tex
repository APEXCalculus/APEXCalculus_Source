\begin{tikzpicture}
\begin{axis}[width=\marginparwidth+25pt,tick label style={font=\scriptsize},minor x tick num=1,axis y line=middle,axis x line=middle,ymin=-.1,ymax=3.2,xmin=-.1,xmax=2.2,name=myplot]
\addplot [{\colorone},smooth,thick] coordinates {(0,1) (2,3)};
\fill[white,draw=black,thick] (axis cs:1,2) circle (1.5pt);
%\draw[thin,dashed,{\colortwo}] (axis cs:0,1.5) -- (axis cs:2.25,1.5);
%\draw[thin,dashed,{\colortwo}] (axis cs:0,2.5) -- (axis cs:6.25,2.5);
%\draw (axis cs:-.1,1.75) node [right]{\tiny$\left.\rule{0pt}{7.5pt}\right\}\epsilon = .5$};
%\draw (axis cs:-.1,2.25) node [right]{\tiny$\left.\rule{0pt}{7.5pt}\right\}\epsilon = .5$};
%\fill[{\colortwo}] (axis cs:2.25,1.5) circle (1pt);
%\fill[{\colortwo}] (axis cs:6.25,2.5) circle (1pt);
%\draw (axis cs:4,1) node [text width = 80pt,align=center] {\footnotesize Choose $\epsilon>0$. Then ...};
\end{axis}
%\fill[{\colortwo}] (1,1) circle (1pt);
\node [right] at (myplot.right of origin) {\scriptsize $x$};
\node [above] at (myplot.above origin) {\scriptsize $y$};
\end{tikzpicture}
% this is the sqrt[x] on [0,5]

